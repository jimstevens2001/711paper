\section{Simulation Design}

The basic design of our simulation is a discrete event simulation
that models the arrivial, departure, and interactions of nodes in a
BitTorrent swarm. The two main entities in the simulation are nodes
and events. A node represents a single user in a BitTorrent swarm.
An event object represents an action that occurs at a specific time.
The file being shared in the swarm is represented simply by the total
number of equal sized pieces and the size of each piece. Pieces are
indexed starting at zero.

Nodes are stored in a global nodes table to allow easy reference by
any event handler or other node. This simulates the BitTorrent tracker
for the swarm and is how nodes initially connect with each other. Each
node contains the state specific to that node, such as the download
progress, the subset of the overall nodes that the node has currently
peered with (meaining it may trade pieces with that peer), and various
other information. The node objects do not emulate everything that takes
place in a normal BitTorrent node, but merely attempt to model at a
level of detail great enough to understand how BitTorrent swarms spread
a file. For example, the files are merely represented by dictionaries
that represent the block sizes of the file, not the actual contents.
Additionally, the low level details of BitTorrent connections, such
as individual have and interest messages (and their timings), are not
directly emulated, but rather emulated via data structures that can be
accessed by the nodes when necessary.

The node state is modified when event handlers execute.
A global work queue is populated with events that are
then sorted according to the schedule times. When events are removed,
the event handler table dispatches the event object to the appropriate
event handling function. Events are represented with lists in which the first
item is the time the event occurs, the second item is the event type, and the remainder
of the list is the event specific parameters for the event handlers (e.g. node IDs).
An example event is [39, 'ADD\_NODE', 3], which represents node 3 joining the swarm at
time 39.

There are several types of events that can occur in the simulation. The ADD\_NODE event simulates a new
node joining the swarm.  The REMOVE\_NODE event simulates a node departing
from the swarm. The EXCHANGE\_ROUND event simulates the unchoking process
that occurs every 10 seconds in a BitTorrent swarm. The first thing the
handler does is make sure the node has enough peers. The unchoke algorithm
then executes next. First, the unchoke algorithm takes the current down
set (or the current up set if the node is a seeder) and sorts based on
the speed of each connection.  It will pick the top four of the up to
five unchoked nodes to remain unchoked. If the optimistic unchoke node
was selected as one of the top four or if the optimistic unchoke node
has expired (which occurs after 30 seconds) or has become uninterested
in current node's pieces, then a new optimistic unchoke node must be
selected. The optimistically unchoked node is selected randomly among
the choked peers. Due to the specification in \cite{bep003}, new peers
(which we define as the three most recent peer connections) are 3 times
more likely than other peers to be optimisically unchoked.  After the
unchoke algorithm completes, the piece uploads and downloads for the node
are scheduled for the duration of the next round. This works by simply
scheduling pieces based on the ability to fulfill the peer's requests
(i.e. the node has a piece that the peer is interested in) in the current
up/down sets until the available bandwidth is allocated. If a piece
will finish before the next EXCHANGE\_ROUND event, then a FINISH\_PIECE
event is scheduled. Last, the next EXCHANGE\_ROUND event is scheduled.
The FINISH\_PIECE event simulates when a piece upload completes between
peers and allows the next piece upload to commence. The LOG event is
used to monitor the state of the simulation and collect results.

The complete implementation, written in Python, of our simulator can be found in our GitHub repository \cite{github}.
