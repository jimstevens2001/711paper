\begin{abstract}
The BitTorrent protocol is a popular peer-to-peer algorithm that is able
to transfer large files efficiently over the internet. A core component
of the algorithm is the local rarest first piece selection algorithm,
which selects pieces to download based on rarity with a node's peer
set to maximize the number of copies of each pieces in a BitTorrent swarm. 
In this paper, we study how good the local knowledge within the peer set is compared to
perfect global knowledge of piece rarity. The hypothesis is that piece
selection with perfect global knowledge can either improve torrent
performance and/or extend the life of unhealthy swarms.  We implemented a
BitTorrent simulator that allows nodes to access perfect global knowledge
at any time and ran experiments. We find that for healthy swarms with seeders
that perfect global knowledge does not improve the average download
times for nodes. However, for unhealthy swarms that have no seeders
but still have a complete copy of the file distributed in the swarm,
we find that perfect global knowledge can help prevent the swarm from
losing pieces, which results in nodes being unable to complete their downloads. 
Next, we analyze a gossip protocol that can approximate
global knowledge efficiently. We find that [INSERT RESULT HERE]. We also
describe an alternative gossip protocol to help seeders determine if it is
safe to leave a swarm without causing pieces to be lost. 

\end{abstract}
