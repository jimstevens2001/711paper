\section{Related Work}

In this section we review the prior works related to our project. We
review three types of work: overviews of BitTorrent's local rarest first
(LRF) piece selection algorithm, measurement studies that analyze the
behavior of the BitTorrent protocol in a variety of real-world situations
and prior attempts to integrate gossip protocols into BitTorrent. These
three categories of prior work influenced the primary contributions
of our project which involved both measuring the behavior of the LRF
algorithm to determine its performance in different situations and
modifying the BitTorrent protocol to incorporate gossip to provide
additional information to the LRF algorithm.

When Cohen introduced BitTorrent he wrote a paper entitled "Incentives
build robustness in BitTorrent" to explain several of the central
design decisions that were made during the development of the protocol
\cite{cohen:1}. Specifically, Cohen explained that BitTorrent has been designed to be
robust through the use of tit-for-tat to determine upload rates,  careful
peering with choking and rarest first piece selection to determine what
piece to request next. The objectives of that paper differ from our work
in that it is primarily meant to explain and justify the initial design
decisions regarding the BitTorrent protocol. Our goal, however,  is to
determine if the rarest first piece selection of the original protocol
works well in all situations and whether or not gossip can be used to
improve its performance by providing additional information. Some of
the initial design decisions have been shown to be less than ideal by
additional research \cite{levin} \cite{bittyrant} and  we feel that the protocol's reliance
on a local view of the swarm in order to determine the next piece to
request is similarly sub-optimal. Our work verifies that the current
rarest first selection algorithm performs poorly in certain situations
and extends the protocol described in Cohen's work to utilize a gossip
protocol thereby allowing the next piece selection to be made in the
light of a global view of the state of the swarm. This extension of the
protocol has the potential to greatly improve the longevity of swarms
which use the BitTorrent protocol as swarms a likely to encounter a
situation in which the current piece selection algorithm performs poorly
at some point during their lifetime.

The work done by Izal et al. \cite{izal} shows the performance of the BitTorrent
protocol in large-scale real-world swarms over the course of a five
month period and highlights several interesting behaviors such as
flash crowds and the large number of peers who do not complete the
download. The measurements presented in this paper confirm that the
BitTorrent protocol performs well under the common situations and the
authors conclude therefore that the protocol is highly effective in
its current form. However, what is more interesting is not necessarily
the most common situations but the extreme situations such as very old
swarms or small swarms or swarms with very high churn rates. These extreme
situations are important to consider because they often contribute must
directly to the death of a swarm and the loss of the data contained
in that swarm. Therefore, instead of studying large-scale steady-state
swarms, we study the effect of more extreme conditions on the health of
swarms in order to better understand how the protocol behaves in these
situations. Through better understanding of the behavior of the protocol
in extreme situations adjustments can be made which can improve the
torrents survivability thereby making the protocol even more effective.

Another measurement project \cite{pouwelse} to determine the behavior of real-world
bit torrent swarms was carried out by Pouwelse et al. which observed
flash-crowd behavior similar to what Izal et al saw but also found
interesting relationships between the health of a torrent and the number
of seeds. Specifically, Pouwelse found that sometimes it can quite long
before there are any more seeds than the original seeder (up to 3 days
in some of the observed cases), Interestingly, Pouwelse also found
that the number of seeds did not seem to determine the lifetime of a
torrent. These findings show that the lifetime and health of a torrent
depend on more complex factors than merely having an appropriate number
of seeds. This lends support to our hypothesis that the behavior of the
protocol, such as poor next piece selection, in certain situations has a
significant effect on the health of the swarm.This work analyzed a much
broader range of torrents than Izal et al and so provides useful data on
the behavior of BitTorrent swarms under more extreme conditions. However,
while this study examines the kinds of swarms that are of interest to
our project, it was primarily interested in the general behavior of the
protocol whereas we are primarily interested in only the relationship
between the information each peer has about the swarm and the correctness
of its next piece selection.By focusing on a more specific aspect of
the behavior of the protocol, our work is able to more clearly identify
areas for improvement and accurately evaluate solutions for that specific
problem. Our work focused on trying to identify the behavior of the
rarest first piece selection of swarms in extreme conditions in which
their local view of the swarm might not be ideal in order to determine
if this might be partially responsible for death of such swarms. As a
result of the measurements taken by this project and others the general
behavior of the BitTorrent protocol in a wide variety of real-world swarms
is now well understood. However, the behavior of the various components
of the protocol and how they contribute to the overall behavior has not
be studied to the same extent. By more closely examining the behaviors
of the constituent algorithms of BitTorrent additional areas requiring
improvement may be identified.

In a more targeted measurement study, Legout et. al. \cite{legout:1} showed
that the rarest first algorithm is effective for distributing rare
pieces and getting the torrent from the transient state (in which
rare pieces exist) to the steady state (in which rare pieces do not
exist) as quickly as the initial seed can upload. This paper assumes
a large peer set and seeders that stay online, while we assume a
smaller peer set and higher churn rate for seeders. This paper also
assumes a goal of keeping leechers interested in each other while we
are focused on torrent longevity. We believe that it is very common
for peers to exit the torrent as soon as their download completes in
real world torrents and this paper does not address that. We create
a mechanism for distributing global knowledge in the torrents to
know what are the true rarest blocks in the swarm, which this paper
cannot do. With global knowledge, the rarest piece algorithm can be
even more effective to ensure that pieces are selected that maxmimize
torrent longetivity.  Bharambe et. al. \cite{bharambe} created a simulator to
study very large torrents. The main contributions of this paper is
that the local rarest first is very effective for large torrents even
if each node only has a small number of peers. They also demonstrated
that LFR does poorly (meaning it fails to provide a high number of
interesting peer connections) when leechers with differing amounts
of file completion are present. Unlike (rarest pieces are enough),
they explicitly assumed that leechers will leave as soon as the
download is complete. Unlike us, we are also assuming that seeds
leave after they have uploaded a copy. This paper mentions the
unfairness that can occur and how seeds and high bandwidth leechers
end up uploading significantly more than they download. We are
curious what happens when these high-value seeds disappear and how
it effects longevity. Their solution is simply that seed bandwidth
should be conserved (e.g. using superseeding) while we attempt to
improve the global knowledge using gossip. Gossip can potentially
provide a more accurate picture of the global state and solve issues
this paper showed such as the leechers with different percentages of
the file.  There is also previous work in applying gossip protocols to
BitTorrent, although none of these systems address the issue we are
concerned about in our work: improving global knowledge for rarest
pieces. The Tribler BitTorrent \cite{tribler} client creates an extension
that allows the gossiping of .torrent files held by other peers
to allow for decentralized keyword searching for content. Qureshi
\cite{proximity} attempts to extend BitTorrent to build a locality-based overlay
rather than the current random overlay. Gossip messages are used to
communicate network coordinates computed from measured round-trip
times between the peers. Guerraoui et. al. \cite{freerider} attempt to combat
techniques that take advantage of altruism in BitTorrent by using
a gossip protocol to detect free riders. Finally, many BitTorrent
clients support peer exchange (PEX), which allows them to exchange
(i.e. gossip) peers in a decentralized fashion \cite{pex}.
