\section{Gossip Protocol}

\subsection{Design}

The goal of the gossip protocol is to extend the lifetime of the swarm
by providing nodes with the knowledge they require in order to make
next piece selections that will efficiently replicate the critically
rare pieces in the swarm. To acoomplish this the gossip protocol
provides each node with a close approximation of omniscient knowledge
about the current state of the swarm. As we saw in Section 4.3
omniscient knowledge allows nodes to make the most intelligent
decision about which piece to download next. 

To create this approximate omnisient knowledge two things have to be
communicated between the nodes in the swarm. First, every node needs
to know each node's local view of the swarm in terms of the pieces
that it and its neighbors hold. This allows the node to determine
accurately which pieces are truely rare in the swarm as a whole. Second, every
node needs to know what each node is planning on downloading
next. This is needed so that a node will not request the same piece as
its neighbors thereby losing an opportunity to replicate a rare
piece. To communicate this information short gossip messages are
formed by each node every exchange round and sent to all of that
node's peers. To form this message, the peer determines how many of
its peers hold each piece in the torrent. This constitues the first
part of the required information for omniscient nodes. Then the node
adds one to the count for every piece that it is requesting from its
peers this exchange round. This constitutes the second part of the
required information for omniscient nodes. From these counts the node
is able to form a list of the pieces ordered according to rarity,
pieces with the lowest count are rarist. It then sends the top 3 rare
pieces to all of its peers in a gossip message with its node id. To
prevent old messages from recirculating throughout the swarm, a
timestamp is also added to the message. Recieving peers will maintain
counts of messages recieved from a particular nodes, if that value is
greater than the message's timestamp then the message is discarded.

At the other side of the message transmission, the recieving node
accumulates all of the gossip messages from its peers and goes
through these messages during its exchange round. Using the lists of
rarest pieces from each message, the node builds a list of globally
rarest pieces. Pieces that are reported as rare by the most nodes are
sought first when the node goes to request pieces for this exchange
round. In this way, nodes efficiently replicate critically rare pieces
thereby extending the lifetime of the swarm. 

In order to provide a information on the entire swarm to every node,
messages are forwarded by a node to all of its peers (except the peer
that sent it). In this way gossip messages are communicated throughout
the entire swarm. As a result of this forwarding the overhead of the
gossip protocol each round is roughly N*N*S where N is the number of
nodes in the swarm and S is the size of the gossip messages.

\subsection{Gossip Results}

To analyze the effectiveness of this gossip protocol we created a
swarm like the ones we used to test the effectivenes of local, global
and omniscient knowledge. The gossip swarm results are presented in
Table X. From these results we can clearly see that the gossip
protocol replicates the critically rare pieces much more efficiently
than local and global knowledge do. This allows it significantly extend the
lifetimes of three of the five test swarms. The gossip protocol does
not currently achieve efficiency on the level of the ideal omniscient
nodes however there is still a great deal of room for improvement
in the protocol. So, with time and fine tuning the effciency may be
improved. Furthermore, the small number of test swarms may have
resulted in the gossip tests simply being unlucky. Omniscient failed
to save one swarm because the conditions of the test swarm made it
impossible to save. It is possible that this simply happened twice for
gossip protocol versus just once for the omniscient nodes in our
limited set of five runs. However, the consistentency with which the
global and local knowledge nodes failed to save the swarm indicates
that this result is reliable.

%Table
%    Runs       | 1 | 2 | 3 | 4 | 5 | 1  | 2  | 3  | 4  | 5  |
%               |  nodes stranded   |     average times      |
%   Gossip      | 11| 99|  1|  7| 99|1981|1254|2293|1901|2130|
%
%               | average nodes stranded  |   average time   | 
%   Gossip      |          43             |       1912       |

\subsection{Recovery Mode}

An alternative gossip protocol that is orthogonal to the protocol described above is 
what we have dubbed ``Recovery Mode'' or ``Emergency Mode''. Basically, we
want to prevent a situation in which a seeder leaves and takes unique pieces
with it, leaving the other nodes unable to finish. This gossip protocol obviously
cannot force seeders to stay, but it can help seeders learn when it is safe to
leave without stranding the rest of the torrent. 

There are many ways such a protocol could be structured. We wanted the lowest
overhead protocol possible, so we chose a method based on a simple OR operation
at each node. Basically, each exchange round a node will check its peers bitfields
and make sure that all of the pieces of the file are available within its peer
set. If not, the node will broadcast a WAIT message to all of its peers. In addition, it will
also broadcast a WAIT message to all of its peers if any of those peers sent a 
WAIT message since the last exchange round. So, the condition to send a WAIT message 
can be thought of as an OR operation. To prevent cycling, a timestamp is placed
on WAIT messages and each node will keep track of the most recent timestamp they have seen. 
Nodes will then only forward messages if the message's timestamp is newer than their most 
recent observed timestamp.

Similar to the leechers, seeders will also receive and forward the WAIT messages. If the 
seed is altruistic enough to want to prevent the torrent from dying when it leaves,
then the seed will send a LEAVE message to its peers when it wants to leave. The LEAVE
message tells the peers to no longer take into account the seeder's pieces when computing
whether or not to send a WAIT message. After the LEAVE message is sent, the seeder will
wait for an additional exchange round and make sure that it receives no new WAIT messages
generated by either its peers or another node in the network. When it receives no WAIT
messages for an exchange round, the seeder is then free to leave without putting the 
swarm in jeopardy of failing. 

The overhead for this protocol is much lower than the previous gossip
system because each node will only send a single message to its peers
each round, so the total number of messages per 10 seconds is PxN. Therefore,
this protocol is a low-overhead mechanism for allowing seeders to know when it is safe
to leave. One could imagine a GUI for a BitTorrent client could have an indicator widget
to alert the end user for the seeder node if WAIT messages are being received or not. As
mentioned above, the recovery mode is orthogonal to the previous gossip
protocol and therefore both could be run at the same time within a swarm.
