\section{Gossip Protocol}

Based on the results of our standard BitTorrent swarm study, we have
decided to focus our gossip protocol on interest messages to allow
nodes beyond the current set of peers to know if a node is looking for
a particular piece. This is particularly helpful for the situation in
which a node joins the swarm with a partially completed set of pieces
and has a difficult time finding peers with the rest of the pieces it
needs.

Note: We are still in the process of implementing the Gossip system, so
we cannot complete this section at the current time.

\subsection{Recovery Mode}

Another gossip protocol that is orthogonal to the protocol described above is 
what we have dubbed ``Recovery Mode'' or ``Emergency Mode''. Basically, we
want to prevent a situation in which a seeder leaves and takes a unique pieces
with it, leaving the other nodes unable to finish. This gossip protocol obviously
cannot force seeders to stay, but it can help seeders learn when it is safe to
leave without stranding the rest of the torrent. 

There are many ways such a protocol could be structured. We wanted the lowest
overhead protocol possible, so we chose a method based on a simple OR operation
at each node. 

Basically, each exchange round a node will check its peers bitfields
and make sure that all of the pieces of the file are available within its peer
set.

If not, the node will broadcast a WAIT message to all of its peers. These peers will
then forward the message to their peers at their next exchange round. A timestamp
is sent with the message to ensure the message do not cycle. 


