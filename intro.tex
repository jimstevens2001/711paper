\section{Introduction}

The BitTorrent protocol is a peer-to-peer algorithm that allows for the
efficient distribution of large files over the Internet while minimizing
the upload requirements for any single node in the network \cite{bep003}. BitTorrent
has become so popular that its traffic currently makes up the plurality
of traffic on the Internet \cite{istudy08}. The protocol works by dividing a file
up into a set of blocks and then sharing the blocks among the nodes in
the network. Each node downloads blocks by selecting a subset of the
network's nodes as peers and then selecting a block to download based on
which block is the rarest among its peers. This rarest-first algorithm
intends to maximize the number of copies of each block in the file.


However, despite their popularity among users, these torrents
often have very limited lifetimes. A common pattern is for a node
to download a file and then exit when finished (although users are
encouraged to upload at least as much as they download). This problem
is even worse for torrents with less popular data. A particularly
common problem is that when a file is almost complete, the last
few blocks are unavailable. Previous work has claimed that the
rarest-first algorithm does good enough compared to more complex
alternatives such as source coding \cite{legout:1}. However, the proliferation
of requests for "re-seeding" on the Internet provide evidence that
additional work may be necessary to reduce the last block problem. We
are particularly interested in torrents that have lost their seeds,
resulting in local views of rarest blocks not accurately reflecting
the global state. If the local view is indeed incorrect, this can
make the torrent's performance worse and possibly shorten the lifetime
of the torrent when the torrent could otherwise recover.


Our work to address the last block problem proceeded in a series of stages
that first established the validity of our hypothesis and then evaluated
the effectiveness of our solution. The initial challenge was to determine
to what degree the last block problem exists. Furthermore,  it was also
necessary to confirm that the lack of global knowledge at each node was
responsible at least in part for the loss of important blocks. After our
hypothesis regarding the cause of the last block problem was validated,
the next step was to determine if gossip could be used to effectively
remedy the situation. To accomplish this we first had to characterize
the types of real-world swarms that typically suffer from the last
block problem. Finally, we used that knowledge and a client modified to
communicate information on the swarm via gossip to show that gossip could
be used to reduce the occurrence of the last block problem significantly.


In order to determine to what degree the problem exists we created a BitTorrent
protocol simulator to record the local view of nodes over time. This data was then
analyzed to compute the global view of rarest blocks. As we suspected,
we found that the rarest block is not necessarily known throughout the
entire torrent. Furthermore, the data showed that sometimes this lack
of knowledge leads to inappropriate next block selection which causes
important blocks to be lost. To address the inadequacy of the local view,
we implemented a gossip protocol to construct a global view of
the swarm at each peer. For the purposes of this work, the global view
only consists of a list of the globally rarest blocks. This reduced the
amount of information that each peer had to store and update. The next block
selection routine was then modified to leverage the information
about the globally rarest blocks to request the blocks most in need
of replication from its neighbors. The effectiveness of this approach
was then verified by running a series of simulations that
approximated a range of swarms.


The primary insight of this work is that the lack of a global view
is a contributing factor to the last block problem in transient
swarms. We also observed that seeds, in addition to peers, have a high
churn rate in real world swarms as users rarely continue to upload for
very long after they complete the download. Thus, many more real world
torrents are in a transient state than was previously thought. The
research community has long considered the local rarest first block
selection routine to be sufficient due to fact that it works well for
steady state swarms and the majority of swarms were assumed to be in
a steady state. However, our insight that seeds have a high churn rate
led us to conclude that the current solution was not sufficient since
it does not work well for swarms that are in a transient state. The fact
that many more swarms are in a transient state than previously considered
may lead to additional investigations of other aspects of the BitTorrent
protocol that only work well under steady state conditions. Furthermore,
the use of gossip to communicate information about the swarm as a whole
allowed us to do efficiently what had previously been considered too
costly to implement \cite{cohen:1}.


The remainder of this paper is organized as follows. Section II discusses
the related work. Section III describes our simulation design.
Section IV is a study of local knowledge vs. global knowledge in BitTorrent.
Section V describes our gossip protocol. Section VI
shows the experimental results of the gossip protocol in action. Section VII
concludes the paper.
