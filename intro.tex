\section{Introduction}

The BitTorrent protocol is a peer-to-peer algorithm that allows for
the efficient distribution of large files over the Internet while
minimizing the upload requirements for any single node in the network
\cite{bep003}. BitTorrent has become so popular that its traffic currently
makes up the plurality of traffic on the Internet \cite{istudy08}. The
protocol works by dividing a file up into a set of blocks and then
sharing the blocks among the nodes in the network. Each node downloads
blocks by selecting a subset of the network's nodes as peers and then
selecting a block to download based on which block is the rarest among
its peers. This rarest-first algorithm intends to maximize the number
of copies of each block in the file.


However, despite their popularity among users, these torrents often have
very limited lifetimes. A common pattern is for a node to download a file
and then exit when finished (although BitTorrent etiquette encourages users to upload at
least as much as they download). This problem is even worse for torrents
with less popular data. A particularly common problem is that when a file
is almost complete, the last few blocks are unavailable. Previous work
claims that the rarest-first algorithm does good enough compared to
more complex alternatives such as source coding \cite{legout:1}. However,
the proliferation of requests for ``re-seeding'' on the Internet provide
evidence that additional work may be necessary to reduce the last block
problem. We are particularly interested in torrents that have lost
their seeds, resulting in local views of rarest blocks not accurately
reflecting the global state. If the local view is indeed incorrect,
this can make the torrent's performance worse and possibly shorten the
lifetime of the torrent when the torrent could otherwise recover.


%Our work to address the last block problem proceeds in a series of stages
%that first establishes the validity of our hypothesis and then evaluate
%the effectiveness of our solution. The initial challenge was to determine
%to what degree the last block problem exists. Furthermore,  it was also
%necessary to confirm that the lack of global knowledge at each node was
%responsible at least in part for the loss of important blocks. After our
%hypothesis regarding the cause of the last block problem was validated,
%the next step was to determine if gossip can effectively
%remedy the situation. To accomplish this we first had to characterize
%the types of real-world swarms that typically suffer from the last block
%problem. Finally, we use that knowledge to communicate information on
%the swarm via gossip to show that gossip can reduce the
%occurrence of the last block problem.


In order to determine to what degree the problem exists we create a
BitTorrent protocol simulator to record the local view of nodes over
time. We then analyze the data to compute the global view of rarest
blocks. We study two types of torrent swarms: swarms with seeders and
swarms that have lost their seeders (but still have a complete copy of
the file distributed within the swarm). For the swarms with seeders,
we found that even a perfect global view does not make a significant
difference in terms of average download time for the nodes. For swarms
without seeders, we found that the rarest block is not necessarily
known throughout the entire torrent and that sometimes this lack of
knowledge leads to inappropriate next block selection which causes
important blocks to be lost as completed nodes leave. To address
the inadequacy of the local view, we implement a gossip protocol to
construct an improved global view of the swarm at each peer.  We then
modify the next piece selection routine to leverage the information
about the globally rarest blocks to request the blocks most in need of
replication from its neighbors. We verified the effectiveness of this
approach by running a series of simulations that approximate a unhealthy
swarms. We also analyze the gossip protocol's overhead to show that gossip
is practical in terms of network usage and node storage.  Additionally,
we also propose an alternative low-overhead gossip protocol to assist
seeds that are altruistic enough to want the swarm to survive after they
leave by letting these seeds know when it is safe to leave.

The primary insight of this work is that the lack of a global view is
a contributing factor to the last block problem in unhealthy swarms
and can be improved with a global picture of the swarm.  Furthermore,
the use of gossip to communicate global information about the swarm 
allowed us to do efficiently what had previously been considered too
costly to implement \cite{cohen:1}.

The remainder of this paper is organized as follows. Section II discusses
the related work. Section III describes our simulation design.  Section IV
is a study of local knowledge vs. global knowledge in BitTorrent.  Section
V describes our gossip protocol and a study of its performance. Section
VI discusses future work and Section VII concludes the paper.
