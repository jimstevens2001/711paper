\section{Future Work}

There are three main areas of future work for this project: adding realism 
to the simulation, deploying the gossip protocol in a BitTorrent client,
and exploring further uses for a low-overhead gossip protocol. 

Currently, our simulation implements all of the core features of
BitTorrent necessary to study the utility of global knowledge and
gossip. The only feature that might affect performance that we did not
implement was super-seeding, which we felt would not change our overall
results. We would like to investigate running swarm simulations with
more realistic features though such as a realistic churn rate and a more
diverse set of downloaders using a real distribution of common BitTorrent
download rates.

We were originally interested in implementing gossip in the Vuze BitTorrent client
and running a simulation in PlanetLab. We decided against that because our simulator
gives easy access to global knowledge to evaluate the upper bound of performance improvement
(which turned out to be very important to our results). Now that we have found a practical
application of gossip in saving unhealthy swarms, we would like to go back and implement
the protocol in a real BitTorrent client and see how this does with real swarms.


Bram Cohen's original BitTorrent paper said that global knowledge is too expensive to acquire,
but we have shown that we can acquire a reasonable approximation of global knowledge
for certain types of knowledge in BitTorrent. Therefore, we would like to investigate
further appliations of improved global knowledge in BitTorrent. We are particularly interested in 
finding misbehaving nodes (e.g. identifying BitThief clients by gossiping about optimistic unchokes)
or other ways to improve performance (e.g. gossiping about the download rates achieved by nodes).
One complicating factor is that many possible uses of gossip also create interesting incentives
that need to be analyzed.

