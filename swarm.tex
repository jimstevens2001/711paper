\section{Standard BitTorrent Study}

\subsection{Experimental Setup}

In this section, we will study two types of swarms and compare the
performance of each when using local knowledge or global knowledge for
the rarest piece selection algorithm. The goal of this section is to
explore the potential for a gossip algorithm to improve BitTorrent.
The first type is a swarm that starts with a single seed and a number
of leechers.  The seed is altruistic and therefore will stay the entire
time. This simulates a ``healthy'' swarm that will not die, but may benefit
from better replication of pieces. The second type is a swarm with no
seeders, although a complete copy of the file exists within the swarm
between a few nodes. This simulates a situation in which the seeders
have left and a swarm is in an ``unhealthy'' state. None of the nodes
are altruistic and therefore leave as soon as possible. For this swarm,
there are two goals. The primary goal is to simply make sure the swarm
survives, because it is possible for a node to leave with pieces that
only it holds, which means no other node can complete. The second goal
is, if the swarm survives, can global knowledge improve performance.

For all of the experiments below, we are assuming the swarm is sharing a
40 MB file divided into 1000 pieces. All leechers have a download rate
selected from a uniform distribution between 50 and 100 KB/s, which is
approximately what a DSL user can achieve in a typical swarm. Finally, all
leechers will join at a random time in the first 100 seconds of the swarm.

The measurements taken for each experiment are simply the download times for
each node in the swarm. We considered other methods such as metrics to
evaluate local vs. global knowledge, such as a distance metric comparing the 
list of pieces sorted by rarity within the peer set vs. rarity globally, but
decided that such metrics were fairly artibtrary and the node download times
tell us what we actually care about: will global knowledge improve a swarm's
performance?

\subsection{Swarms with Seeders}

In this section, we will examine swarms with a single seeder at the beginning
and a number of leechers that join in the first 100 seconds. To ensure
that a diverse set of conditions are studied, the number of nodes in the swarm, 
number of peers per node, and the initial seeder's upload rate will be varied.
In the healthiest swarms with a seeder, there should be many nodes, many peers per node, and
a high seeder upload rate. Likewise, the least healthy swarms have fewer nodes,
fewer peers per node, and a lower upload rate for the seeder. However, if too fewer
nodes exist than the peers allowed per node, then each node will have perfect global
knowledge by default, therefore, the number of nodes must always be kept above
the number of peers per node. 

% Variables:
% Number of nodes: 100, 200
% Number of peers: 10/20, 20/30, 30/40 (desired/max)
% Seed upload rate: 5, 20, 40, 100

% Experiments (local vs global for each):
% (100, 10, 100) L = 1780, 1533 G = 2040, 1734
% (100, 20, 100) L = 1650, 1345 G = 1700, 1429
% (100, 30, 100) L = 1550, 1396 G = 1720, 1368

% (150, 20, 100)
% (200, 30, 100) L = 1950, 1478 G = 2060, 1565

% (100, 10, 20)
% (100, 10, 40) 

% Also, do a couple runs with omniscient

% Table 1:
% nodes, peers, rate, local total runtime, local average download time, global total runtime, global average download time

% Figure 1:
% Have a column of graphs for a couple experiments, show local on left, global on right
% This will show all download times vs. download rates


\subsection{Swarms with No Seeders}

Describe the NO\_SEED test and give results.
